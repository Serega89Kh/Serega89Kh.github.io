% "Лабораторная работа 2"

\documentclass[a4paper,12pt]{article} % тип документа

% report, book

%  Русский язык

\usepackage[T2A]{fontenc}			% кодировка
\usepackage[utf8]{inputenc}			% кодировка исходного текста
\usepackage[english,russian]{babel}	% локализация и переносы


% Математика
\usepackage{amsmath,amsfonts,amssymb,amsthm,mathtools} 
\usepackage{wasysym}
\usepackage{hyperref}

%Заговолок
\author{ИИТиТО}
\title{Особености технологии набора технического текста{}}
\date{\today}


\begin{document} % начало документа
\maketitle
\newpage
\section{Задание 1}

\subsection{Для чего предназначена издательская система \LaTeX?}

\begin{flushleft}

\TeX  — это издательская система, предназначенная для набора научно-технических
текстов высокого полиграфического качества.\LaTeX  — один из наиболее популярных
макропакетов на базе \TeXа, существенно дополняющий его возможности. \LaTeX 2 —
его последняя версия, которая по праву считается наиболее удачным расширением TEXа.
MiKTeX — это свободно распространяемая реализация \TeX  под Windows, включающая
в себя практически все известные расширения.
Создаваемые с помощью \LaTeXа тексты могут содержать математические формулы,
таблицы и графические изображения. Поддерживается автоматическая нумерация страниц, разделов, формул и пунктов перечней. Система сама генерирует оглавление, списки
таблиц и иллюстраций, перекрёстные ссылки, сноски, колонтитулы и предметный указатель. Наконец, имеется возможность определять собственные макрокоманды и стили.
\end{flushleft}
\begin{center}
\subsection{В каких случаях рационально использовать \LaTeX?}
\end{center}

\LaTeX позволяет писать хорошо структурированные документы. Но в нем сложно и долго создавать полностью новый макет, следовательно трудно создать небольшие документы. Может использоваться для верстки\footnote{Верстка -- это расположение составных элементов (текста, заголовков, изображений, таблиц) на странице документа. } академических, летературных текстов и нотных грамот.\\

\subsection{Какие преимущества имеет работа в этот системе?}

Преимущества LaTeX для академического использования состоят в том, что он производит разумно свёрстанные документы, которые хорошо выглядят именно в таком виде, в каком представители научных кругов обычно любят публиковать документы.

\subsection{Какие сложности могут возникнуть при работе в этот системе?}

Сверстать документ так, чтобы его было приятно и удобно читать – это далеко не такая простая задача. Пакет \LaTeX позволяет получить приемлемый результат за разумный промежуток времени без необходимости привлечения специалиста-верстальщика. Однако для создания сложных текстов всё-таки потребуется потратить время на изучение возможностей /\LaTeX. К счастью, для \LaTeX присутствует множество информации, как по самому пакету, так и по его классам и стилям.
\begin{itemize}
\item Неопытному пользователю может показаться слишком громоздким набор из множества команд. 
\item Cложности могут возникнуть при работе с неструктурированными документами.  
\end{itemize}

\subsection{Какие недостатки отмечают пользователи при работе с этой системой?}
\begin{itemize}
\item Готовый результат можно увидеть только после сборки.
\item Набранный текст в  LaTeX есть полноценный программный код. Во время обучения будет очень тяжело найти ошибку.
\item Количество людей которые умеют использовать LaTeX в СНГ невелико. Концентрация может меняться от 0 (в школах и гуманитарных факультетах) до обязательного использования всеми студентами (продвинутые кафедры физических и математических факультетов).
\item Наличие большого количества не очевидных случаев, которые решаются с помощью гугла.
\item Требуется потратить от недели до нескольких месяцев на обучение.
\end{itemize}

\section{Задание 2}

\subsection{Какая основная цель написания текста?}

Передать читателю идеи, информацию или знания.

\subsection{Что такое абзац?}

Отрезок письменной речи, состоящий из одного или нескольких предложений.

\subsection{Что делать, если сбилась кодировка?}

Если вы пишете многоязычный документ с конфликтующими входными кодировками, можно переключиться на Unicode при помощи пакета ucs.

\subsection{Сколько видов тире существует в \LaTeX}

Четыре. Три из них получается различным числом последовательных знаков -.

\subsection{Как напечатать знак многоточия?}

С помощью команды \textbackslash ldots.

\subsection{Зачем в конце предложения \LaTeX вставляет интервал?}

Чтобы сделать текст более читабельным.

\subsection{Как правильно оформить сноски?}

Сноски всегда должны помещаться после слова или предложения, к которым они относятся. В русском языке сноски, относящиеся к предолжению, должны ставиться сразу перед точкой или запятой.

\subsection{Как выделить важные слова в \LaTeX?}

Подчеркиванием. В \LaTeX это команда \textbackslash underline

\subsection{Как создать таблицу в \LaTeX?}

С помощью окружения tabular.

\subsection{Что представляеют из себя плавающие объекты?}

Любая иллюстрация или таблица, не умещающаяся на текущей странице, может `плавать', перемещаясь на следующую страницу в процессе заполнения текстом текущей. Это делается для того, чтобы страницы не были частично пустыми.

\section{Задание 4}

\begin{itemize}
\item \href{https://www.latex-project.org/}{Официальный сайт \LaTeX}
\item \href{https://ru.wikibooks.org/wiki/LaTeX}{Викиучебник по \LaTeX}
\item \href{https://www.coursera.org/learn/latex}{Создание документов и презентаций в \LaTeX}
\item \href{http://www.astronet.ru/db/msg/1202050/mathmode.html}{Математика в \LaTeX}
\item \href{http://www.machinelearning.ru/wiki/index.php?title=LaTeX}{Шпаргалка по работе в \LaTeX}
\end{itemize}
\end{document} % конец документа