% "Самостоятельная работа"

\documentclass[a4paper,12pt]{article} % тип документа

% report, book

%  Русский язык

\usepackage[T2A]{fontenc}			% кодировка
\usepackage[utf8]{inputenc}			% кодировка исходного текста
\usepackage[english,russian]{babel}	% локализация и переносы


% Математика
\usepackage{amsmath,amsfonts,amssymb,amsthm,mathtools} 


\usepackage{wasysym}

%Заговолок
\author{Храмов Сергей, ИВТ 3.2}
\title{Самостоятельная работа 5}
\date{\today}


\begin{document} % начало документа
\maketitle
\newpage
\section*{Сымов Павел Николаевич}


\parindent=1cm
\hspace*{1cm} Сымов Павел Николаевич является моим прадедушкой по маминой линии. Родился в 1904 году в деревне Абамза, Чувашская АССР. Летом 1942 года записался добровольцем в ряды Красной Армии и попал в 37 Гвардейский Минометный полк Белоруского Фронта. Вместе со своим полком дошел до Берлина. 

26 апреля 1945 года в 23 часа противник подвег его машину сильному артиллерийскому обстрелу. Шофер был контужен и не мог вывести орудие из под огня. Павел Николаевич несмотря на сильный артобстрел вывел шофера гвардии красноармейца Медведева в укрытие и по своей инициативе вывел из под обстрела боевую установку, тем самым спас жизнь шофера и сохранил боевую установку. За совершенный подвиг, 29 мая 1945 года был  награжден Орденом Красной Звезды.

\end{document} % конец документа