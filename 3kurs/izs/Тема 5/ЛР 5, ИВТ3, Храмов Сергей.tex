% "Лабораторная работа"

\documentclass[a4paper,12pt]{article} % тип документа

% report, book

%  Русский язык

\usepackage[T2A]{fontenc}			% кодировка
\usepackage[utf8]{inputenc}			% кодировка исходного текста
\usepackage[english,russian]{babel}	% локализация и переносы


% Математика
\usepackage{amsmath,amsfonts,amssymb,amsthm,mathtools} 


\usepackage{wasysym}

%Заговолок
\author{Храмов Сергей, ИВТ 3.2}
\title{Основы работы в \LaTeX}
\date{\today}


\begin{document} % начало документа
\maketitle
\newpage
\section*{Издательские системы}


\parindent=1cm
\hspace*{1cm} Настольные издательские системы (НИС) — это программы, предназначенные для профессиональной издательской деятельности, позволяющие осуществлять электронную верстку широкого спектра основных типов документов.

Предусмотренные в программных пакетах данного типа средства позволяют:
\begin{itemize}
\item компоновать (верстать) текст;
\item использовать всевозможные шрифты и полиграфические изображения;
\item осуществлять редактирование на уровне лучших текстовых процессоров;
\item обрабатывать графические изображения;
\item обеспечивать вывод документов высокого качества;
\item и др.
\end{itemize}
Известными пакетами среди издательских систем для компьютеров являются PageMaker, QuarkXPress, Scribus и др.

Имеются два основных вида издательских систем. Издательские системы первого вида очень удобны для подготовки небольших материалов с иллюстрациями, графиками, диаграммами, различными шрифтами в тексте (например, газет, небольших журналов). Типичный пример такой системы — Aldus PageMaker.

Издательские системы второго вида больше подходят для подготовки объемных документов, например книг. Одна из таких систем — Ventura Publisher (Corel Ventura) – управляет меню и может считывать тексты, подготовленные с помощью других текстовых редакторов (например, Microsoft Word), сохраняя при этом параметры форматирования, заданные этим редакторами.

Основная операция издательских систем — верстка (размещение текста по страницам документа, вставка рисунков, оформление текста различными шрифтами и т.д.). Редактирование текста в издательских системах менее удобно, чем в текстовых редакторах. Поэтому бывает, что документы готовят в два этапа: сначала набирают текст в текстовом процессоре, а затем считывают его издательской системой и осуществляют окончательную подготовку документа.

Основные функции издательских систем: использование сотен видов шрифтов (начертаний и размеров символов текста), которые отображаются на экране так же, как при печати; изменение и корректировка рисунков и диаграмм; формирование таблиц; выравнивания; работа с формулами и др.

Большинству пользователей для выполнения издательских работ может быть вполне достаточно возможностей текстового процессора, в котором есть элементы цветовыделения и средства графических редакторов.

\subsection*{Издательская система \TeX}


\hspace*{1cm} \TeX  — система компьютерной вёрстки, разработанная американским профессором информатики Дональдом Кнутом в целях создания компьютерной типографии. В неё входят средства для секционирования документов, для работы с перекрёстными ссылками. Многие считают \TeX  лучшим способом для набора сложных математических формул. В частности, благодаря этим возможностям, TeX популярен в академических кругах, особенно среди математиков и физиков.

Название произносится как «тех». В написании буква E опущена ниже T и X. 

\TeX  является свободным ПО.

В отличие от обыкновенных текстовых процессоров и систем компьютерной вёрстки, построенных по принципу WYSIWYG, в \TeX’е пользователь лишь задает текст и его структуру, а TeX самостоятельно на основе выбранного пользователем шаблона форматирует документ, заменяя при этом дизайнера и верстальщика. Документы набираются на собственном языке разметки в виде обычных ASCII-файлов, содержащих информацию о форматировании текста или выводе изображений. Эти файлы (обычно имеющие расширение «.tex») транслируются специальной программой в файлы «.dvi» (device independent — «независимые от устройства»), которые могут быть отображены на экране или напечатаны. DVI-файлы можно специальными программами преобразовать в PostScript, PDF или другой электронный формат.

Ядро \TeX’а представляет собой язык низкоуровневой разметки, содержащий команды отступа и смены шрифта. Огромные возможности в \TeX’е предоставляют готовые наборы макросов и расширений. Наиболее распространённые расширения стандартного \TeX’а (наборы шаблонов, стилей и т. д): \LaTeX  (произносится «латех» или «лейтех») и AMS-TeX. При использовании пакета расширения \LaTeX  можно превратить разросшуюся статью в книгу изменением одного слова в исходном файле, вставлять оглавление одной командой, не задумываться о нумерации разделов, теорем, рисунков. Есть много пакетов для оформления химических формул (например, пакет XyMTeX), диаграмм (xypic), создания презентаций и визитных карточек и тому подобного.

\subsection*{Дональд Кнут}


\hspace*{1cm} Дональд Эрвин Кнут (англ. Donald Ervin Knuth; род. 10 января 1938 года, Милуоки, штат Висконсин) — американский учёный в области информатики, эмерит-профессор Стэнфордского университета, профессор СПбГУ и других университетов, преподаватель и идеолог программирования, автор 19 монографий (в том числе ряда классических книг по программированию) и более 160 статей, разработчик нескольких известных программных технологий. Автор всемирно известной серии книг, посвящённой основным алгоритмам и методам вычислительной математики, а также создатель настольных издательских систем \TeX и METAFONT, предназначенных для набора и вёрстки книг научно-технической тематики (в первую очередь — физико-математических).

Родился в семье преподавателя. Его отец преподавал бухгалтерский учёт, а также занимался печатным делом на дому как любитель (этим можно объяснить последующий интерес Дональда к этому делу и такие разработки как \TeX). С юных лет в нём наблюдалась склонность к математике, физике и музыке.

Окончил с отличием отделение математики Кейсовского технологического института (бакалавр, 1960). Одновременно за значительные достижения в программировании был удостоен степени магистра. Спустя три года получил докторскую степень в Калифорнийском технологическом институте. Преподавал там же математику и одновременно работал консультантом по проблемам разработки программного обеспечения в корпорации Burroughs.

В 1968 году перешёл в Стэнфордский университет. В 1968—1969 годах также работал в Институте оборонных исследований. Приглашённый профессор математики в Университете Осло (1972, 1973). В Стэнфорде под его руководством защищено 28 докторских диссертаций.

Большое влияние на молодого Кнута оказали работы Андрея Ершова, впоследствии его друга.

Поскольку Кнут всегда считал монографию «Искусство программирования» основным проектом своей жизни, в 1993 году он вышел в отставку с намерением полностью сконцентрироваться на написании недостающих частей и приведении в порядок существующих.

\subsection*{Издательская система \LaTeX}


\LaTeX  — наиболее популярный набор макрорасширений (или макропакет) системы компьютерной вёрстки \TeX, который облегчает набор сложных документов. В типографском наборе системы \TeX форматируется традиционно как \LaTeX.

Важно заметить, что ни один из макропакетов для \TeX’а не может расширить возможностей \TeX (всё, что можно сделать в \LaTeX’е, можно сделать и в \TeX’е без расширений), но, благодаря различным упрощениям, использование макропакетов зачастую позволяет избежать весьма изощрённого программирования.

Пакет позволяет автоматизировать многие задачи набора текста и подготовки статей, включая набор текста на нескольких языках, нумерацию разделов и формул, перекрёстные ссылки, размещение иллюстраций и таблиц на странице, ведение библиографии и др. Кроме базового набора существует множество пакетов расширения \LaTeX. Первая версия была выпущена Лесли Лэмпортом в 1984 году; текущая версия, \LaTeX2, после создания в 1994 году испытывала некоторый период нестабильности, окончившийся к концу 1990-х годов, а в настоящее время стабилизировалась (хотя раз в год выходит новая версия).

Общий внешний вид документа в \LaTeX определяется стилевым файлом. Существует несколько стандартных стилевых файлов для статей, книг, писем и т. д., кроме того, многие издательства и журналы предоставляют свои собственные стилевые файлы, что позволяет быстро оформить публикацию, соответствующую стандартам издания.

Во многих развитых компьютерных аналитических системах, например, Maple, Mathematica, Maxima, Reduce возможен экспорт документов в формат *.tex. Для представления формул в Википедии также используется \TeX-нотация.

Термин \LaTeX относится только к языку разметки, он не является текстовым редактором. Для того, чтобы создать документ с его помощью, надо набрать .tex-файл с помощью какого-нибудь текстового редактора. В принципе, подойдёт любой редактор, но большая часть людей предпочитает использовать специализированные, которые так или иначе облегчают работу по набору текста LaTeX-разметки.

Будучи распространяемым под лицензией \LaTeX Project Public License, \LaTeX относится к свободному программному обеспечению.

\subsection*{Лесли Лэмпорт}


Лесли Лэмпорт (англ. Leslie Lamport; 7 февраля 1941 года, Нью-Йорк) — американский учёный в области информатики, первый лауреат премии Дейкстры. Разработчик \LaTeX — популярного набора макрорасширений системы компьютерной вёрстки \TeX, исследователь теории распределённых систем, темпоральной логики и вопросов синхронизации процессов во взаимодействующих системах. Лауреат Премии Тьюринга 2013 года.

Член Национальной академии наук США (2011), Национальной инженерной академии США (1991).

Окончил школу в Бронксе (Bronx High School of Science), степень бакалавра по математике получил в Массачусетском технологическом институте в 1960 году. Степени магистра (1963) и доктора философии (1972) получил в Брандейском университете.

С 1970-х годов работал в Массачусетском технологическом институте, SRI International, DEC и Compaq, с 2001 года — сотрудник Microsoft Research.

Исследования Лэмпорта заложили основы теории распределённых систем. Среди самых его знаменитых работ можно назвать:
\begin{itemize}
\item "Time, Clocks, and the Ordering of Events in a Distributed System". Эта работа получила награду 2000 PODC Influential Paper Award в 2000 г., а в 2007 г. - ACM SIGOPS Hall of Fame Award.
\item "How to Make a Multiprocessor Computer That Correctly Executes Multiprocess Programs", давшая определение последовательной консистентности,
\item "The Byzantine Generals' Problem",
\item "Distributed Snapshots: Determining Global States of a Distributed System" и
\item "The Part-Time Parliament".
\end{itemize}
\section*{Основные правила создания текстового документа.}


В начале любого документа созданого в \LaTeX  мы создаем преамбулу, выглядит она так: \\[5mm]
\textbackslash documentclass[a4paper,12pt]\{article\} - определение типа документа\\ 
\textbackslash usepackage[T2A]\{fontenc\} - определение кодировки\\
\textbackslash usepackage[utf8]\{inputenc\} - определение кодировки исходного текста\\
\textbackslash usepackage[english,russian]\{babel\} - создание локализации и переносов\\[5mm]
Можно указать названия пакетов, которые вы хотите использовать в своем тексте, в моем случае это пакеты математики:\\[5mm]
\textbackslash usepackage\{amsmath,amsfonts,amssymb,amsthm,mathtools\} \\
\textbackslash usepackage\{wasysym\} \\[5mm]
Дальше создаем заголовок:\\[5mm]
\textbackslash author\{\} - собственно имя автора\\
\textbackslash title\{\} - название документа\\
\textbackslash date\{\} - дата создания документа\\[5mm]
Теперь приступаем к написанию текста:\\[5mm]
\textbackslash begin\{document\} \\[5mm]
Текст документа \\[5mm]
\textbackslash end\{document\} \\[5mm]
Рекомендую, сразу после команды \textbackslash begin\{document\} написать 2 команды: \textbackslash maketitile и \textbackslash newpage. Первая команда отражает в вашем документе заголовок, а вторая создает новую страницу.\\[5mm]
Ну а теперь перейдем к основным командам:\\[2mm]
Создание раздела - \textbackslash section\{Название раздела\} - раздел с номером, \textbackslash section*\{Название раздела\} – раздел без номера\\[2mm]
Создание подраздела - \textbackslash subsection\{Название раздела\} - подраздел с номером \textbackslash subsection*\{Название раздела\} – подраздел без номера\\[2mm]
Для создания нового абзаца с красной строки, нужно оставить пустую строчку, без красной строки написать команду \textbackslash \textbackslash \\[2mm]
Для создания вертикального отступа фиксированного размера, например, 2 сантиметров написать \textbackslash \textbackslash [2cm]. Горизонтального отступа \textbackslash hspace\{2cm\}\\[2mm]
Различные видоизменения шрифта: жирный - \textbackslash textbf\{текст\}, курсив - \textbackslash textit\{текст\}, подчеркивание - \textbackslash underline\{текст\}\\[2mm]
Для создания дефиса поставить одну черточку, тире – две черточки.\\[5mm]
Этот минимальный набор команд поможет вам создать небольшой текст, для написания больших и професиональных текстов, советую держать какое-нибудь пособоие, например каталог <<\LaTeX  И ЕГО КОМАНДЫ>> под авторством С.В. Клименко и М.В. Лисиной.
\end{document} % конец документа