% "Лабораторная работа 2"

\documentclass[a4paper,12pt]{article} % тип документа

% report, book

%  Русский язык

\usepackage[T2A]{fontenc}			% кодировка
\usepackage[utf8]{inputenc}			% кодировка исходного текста
\usepackage[english,russian]{babel}	% локализация и переносы


% Математика
\usepackage{amsmath,amsfonts,amssymb,amsthm,mathtools} 
\usepackage{wasysym}
\usepackage{hyperref}

%Заговолок
\author{Храмов Сергей, ИВТ 3}
\title{Наберите и оформите текст лекции по математике{}}
\date{\today}


\begin{document} % начало документа
\maketitle
\newpage

\section{Определенный интеграл}

Неопределённый интеграл для функции f(x) — это совокупность всех первообразных данной функции.\\
Если функция f(x) определена и непрерывна на промежутке (a,b) и F(x) — её первообразная, то есть {F'(x)=f(x) при a<x<b, то:\\

$\int f(x) \cdot \partial x = F(x) + C, a<x<b$, где С — произвольная постоянная.\\

Основные свойства неопределённого интеграла приведены ниже.\\
$\partial (\int f(x) \partial x) = f(x) \partial x$\\
$\int \partial (F(x)) = F(x) + C$\\
$\int a * f(x) \partial x = a * \int f(x) \partial x$\\
$\int (f(x) +- g(x)) \partial x = \int f(x) \partial x +- \int g(x) \partial x$\\

Основные интегралы:
1.$\int 0 \cdot \partial x =C$  \\
2.$\int 1 \cdot \partial x = \int \partial x = x + C$ \\
3.$\int x^{n} \partial x = \frac{x^{n + 1}}{n + 1} + C, n \neq -1, x>0$ \\
4.$\int \frac{\partial x}{x} = \ln|x| + C$ \\
5.$\int a^{x} \partial x = \frac{a^{x}}{\ln a} + C, a > 0$ \\
7.$\int e^{x} \partial x = e^{x} + C  $\\
8.$\int \cos x \partial x = \sin x + C  $\\
9.$\int \sin x \partial x = -\cos x + C  $\\
10.$\int \frac{\partial x}{\cos^{2}x} = \tg x + C  $\\
11.$\int \frac{\partial x}{\sin^{2}x} = -\ctg x + C  $\\
12.$\int \frac{\partial x}{\sqrt{a^2 - x^{2}}} = \arcsin \frac{x}{a} + C $  \\
13.$\int \frac{\partial x}{{a^2 + x^{2}}} = \frac{1}{a}\arctg \frac{x}{a} + C  $  \\

\end{document} % конец документа