% "Лабораторная работа 2"

\documentclass[a4paper,12pt]{article} % тип документа

% report, book

%  Русский язык

\usepackage[T2A]{fontenc}			% кодировка
\usepackage[utf8]{inputenc}			% кодировка исходного текста
\usepackage[english,russian]{babel}	% локализация и переносы


% Математика
\usepackage{amsmath,amsfonts,amssymb,amsthm,mathtools} 
\usepackage{wasysym}
\usepackage{hyperref}

%Заговолок
\author{Храмов Сергей, ИВТ 3}
\title{Особенности технологии создания текста с формулами{}}
\date{\today}


\begin{document} % начало документа
\maketitle
\newpage

\section{Задание 1}

\subsection{Как вывести квадратный корень?}
\textbackslash sqrt

\subsection{Как вывести горизонтальные фмгурные скобки?}
\textbackslash overbrace

\subsection{Как выглядит оператор интеграла?}
\textbackslash int

\subsection{Как выглядит оператор суммы?}
\textbackslash sum

\subsection{Как выглядит оператор произведения?}
\textbackslash prod

\subsection{Можно ли подстраивать пробелы в математических формулах?}
Да, с помощью команды \textbackslash quad

\subsection{Будут ли автоматически разбиваться длинные уравнения?}
Нет, для это нужно самому указать, где их разбить с помощью специальных методов

\subsection{Можем ли мы увидеть "фантомы"?}
Нет

\subsection{Какими командами настраивается размер шрифта в математических формулах?}
\textbackslash displaystyle (123), \textbackslash textstyle (123),  \textbackslash scriptstyle (123) и \textbackslash scriptscriptstyle (123).

\subsection{Как верстать определения, аксиомы, теоремы и законы?}
\textbackslash newtheorem{название}[счетчик]{текст}[раздел]
\section{Задание 2}
1.Вычислить значение функции y(x) для каждого x. Коэффициенты t, k, s являются константами и вводятся с клавиатуры. Значение x находится в интервале [-25;15] и изменяется с шагом 1.\\
\[
{y = t \cdot x^3 + k \cdot x + s}
\]
2.Изменяя значение переменной k (начальное значение k=1, шаг 1), найдите при каком k значение функции y(k) первысит 1200.\\
\[
{y = 2^{k+2} - 5}
\]
3.В данной функции w, n, c-константы, x-вводится с клавиатуры. Найти значение функции.\\
 \[
{y} = \{
\begin{array}{cc}
{w^2, {при }  x \geqslant 1.5} \\
{ n \cdot x + 9, {при } w \in (-12;1.5)}\\
{c-x, {при } x \leqslant -12} \\
\end{array}
 \]
\section{Задание 3}
На рисунке дана функция. Коэффициенты a, b, c являются константами, а x находится в интервале [ -10 ; 18 ] и изменяется с шагом h, значение которого вводится с клавиатуры. \\
Найти все значения функции для заданных x. \\
\[
y = a x^2 + b x+ c 
\] \\
 2.На рисунке дана функция. Найти значение переменной n, при котором значение
 функции превысит 1000.
\[
y = 2^{n-1} + 3
\] 
 3.На рисунке дана функция. В данной функции t,a,s - const, x -- вводится с клавиатуры. Найти значение функции.
 \[
 {y} = \{
\begin{array}{ll}
{t, { при }  x \geqslant 3} \\
{ a \cdot x - s, { при } x \in (-5.5; 3)}\\
{ x^3, { при } x \leqslant -5.5} \\
\end{array}
 \]
\section{Задание 4}

Создание примечаний\\
Для того чтобы сделать примечание к тексту нужно воспользоваться командой \textbackslash footnote{...}, поместив внутри скобок текст примечания.\\
Например,
{\it Примечание} Для того чтобы сделать примечание к тексту нужно воспользоваться командой  \textbackslash verb|\textbackslash footnote{...}|, поместив внутри скобок текст примечания. \footnote{Например, вот этот текст окажется в примечании и будет набран в конце страницы.} Не беспокойтесь относительно нумерации, примечания нумеруются автоматически. \footnote{Хотя добиться, чтобы на каждой странице нумерация начиналась сначала довольно трудно. Во всяком случае, не с этого надо начинать!}\\
2. Библиография\\
Обстановка \\
\textbackslash begin{thebibliography}{99}

\textbackslash end{thebibliography}

служит для печатания списка цитируемой литературы. Ее единственным параметром  является образец текста, ширина которого будет использована для определения размера метки записей, одинакового для всех записей списка.\\
Внутри обстановки записи начинаются с команд \textbackslash bibitem[lab]{key},  где key — ключ, который используется при цитировании в тексте, а  lab — факультативный параметр, задаваемый, если автор не хочет пользоваться автоматически генерируемой сквозной нумерацией.\\
В тексте можно ссылаться на эти записи, используя команды \textbackslash cite{key}, — вместо них автоматически подставятся метки.

\end{document} % конец документа